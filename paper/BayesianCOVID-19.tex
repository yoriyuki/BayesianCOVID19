\documentclass{article}
\usepackage{amsmath}
\title{Bayesian modeling of COVID-19 epidemic: Japanese case}
\author{Yoriyuki Yamagata}
\date{\today}

\begin{document}

\maketitle

\begin{abstract}
    We estimate daily change of a ``contact frequency'' and reporting rate in Japan by a Bayesian method.
    The estimate reveals no consistent trend.
    Although our analysis is preliminary, this may raise questions to the effectiveness of Japanese policy against COVID-19 infection.
\end{abstract}

\section{Introduction}

In the wake of COVID-19 epidemic, Japanese government gradually  employed public health measures against COVID-19.
In the first stage, the stronger quarantine measures at the boarder were implemented.
Once patients which had no connection to Wuhan appeared inside the boarder, The government started to track these patients as far as possible and tried to find people contacted to these patients.
Once these ``track and quarantine clusters'' tactics were overwhelmed by the number of patients, the government started to ask ``behavior changes'' to people, culminating ``declaration of emergency situation'' at April 6th.
Public facilities like libraries were closed.
Large shopping malls and entertainment business such as movie theatres were asked to be closed.
Restaurants are asked to shorten their operating hours and stop providing alcoholic beverages at night.
Working from home was encouraged and the citizens were advised to avoid crowed areas and generally avoid to go outside unnecessarily.

However, Japanese legal system does not have enough mechanism to ``enforce'' these policies.
Thus the effectiveness of these policies are questionable.
In fact, many people are still commuting to their office because many companies lack the necessary ability of allowing their employee to work from home.
Many small restaurants and cafes are sill running business because lack of financial compensation.

In this paper, we apply a Bayesian method to estimate daily changes of a ``contact parameter'' which determines the speed of infection, and reporting rate.
The result does not reveals any significant trend toward to reduction of the infection speed.
Although our analysis is preliminary, this may raise questions to the effectiveness of Japanese policy against COVID-19 infection.

\section{Related works}

Several works employs data-driven methods to predict and measure the public health measure of COVID-19.

Anastassopoulou et al.~\cite{Anastassopoulou2020} applies SIRD model to Chinese official statistics, estimating parameters using linear regression.
Reporting rate is not estimated from data, but simply assumed.
By these models and parameters, they make a prediction of COVID-19 epidemic in Hubai province.

Diego Caccavo~\cite{Caccavo2020} and independently Peter Turchin~\cite{Turchin2020} apply modified SIRD models, in which parameters change overtime following specific function forms.
Parameters govern these function are estimated by minimizing the sum-of-square-error.
However, using the sum-of-square method causes over-fitting and always favors a complex model, therefore it is not suitable to access policy effectiveness.
Further, fitting SIRD model in the early stage of infection is difficult, as pointed out in stat-exchange~\footnote{https://stats.stackexchange.com/questions/446712/fitting-sir-model-with-2019-ncov-data-doesnt-conververge}.
Using a Bayesian method, we avoid these problems in some degree, because a Bayesian method estimates parameter distribution instead of point estimate.
Thus, we can assess the degree of confidence of each parameter.
Further, by well-established statistical methods, we can compare explanatory power of different models.

Flaxman et al.~\cite{Flaxman2020} uses a Bayesian model to estimate the policy effectiveness.
The methodology is different from us, because they assume immediate effects from the policies implemented.
Further, they use a discrete renewal process, a more advanced model than SIRD model.
They use parameters estimated from studies of clinical cases while we use a purely data driven method.

\section{Method and materials}

\subsection{Model}

We use the well-known SIRD model but do some modification.
Bayesian estimation requires large number of simulation runs, so solving ordinary differential equations is too computationally expensive.
Therefore, we replace ordinary differential equations to difference equations.
However this causes numerical instability.
To mitigate instability, we split $\beta$ in SIRD model into a ``contact frequency'' $b$ and infection probability $p$.
Let $S$ be the number of susceptible people, $I$ that of infected people, $D$ that of death and $P$ population.
The probability of that one susceptible individual will be infected for each time step is estimated by $1 - (1 - p)^{b I / (P - D)}$.
We can introduce the ``quarantine measure'' $\alpha$, which indicates the ratio of interaction between infected people and uninfected people, as $1 - (1 - p)^{b \alpha I / (P - D)}$.
However, this cannot be distinguished by the equation above so we use $1 - (1 - p)^{b I / (P - D)}$ as a probability.
Then, the average number of new infection is $S \cdot \{1 -  (1 - p)^{b I / (P - D)} \}$.
Further, the process is stochastic, therefore 

\begin{align}
    NI(t) &\sim \textup{Poisson}(S \cdot \{1 -  (1 - p)^{b I / (P - D)} \})\\
    NR(t) &\sim \textup{Poisson}(\gamma I)\\
    ND(t) &\sim \textup{Poisson}(\delta I)\\
    I(t+1) &=  I(t) + NI(t) - NR(t) - ND(t)\\
    S(t+1) &= S(t) - NI(t)\\
    R(t+1) &= R(t) + NR(t)\\
    D(t+1) &= D(t) + ND(t)
\end{align}

We cannot expect that these values are directly observable, because many (or most) cases are mild or asymptomatic.
Therefore, we introduce the reporting rate $q$ and let the number of cumulative observed cases $C_{\text{obs}}$, recovered $R_{\text{obs}}$ and death $D_{\text{obs}}$ as
\begin{align}
    NI_{\text{obs}}(t+1) &\sim \textup{Poisson}(q * NI(t))\\
    NR_{\text{obs}}(t+1) &\sim \textup{Poisson}(\gamma * I_{\text{obs}}(t))\\
    ND_{\text{obs}}(t+1) &\sim \textup{Poisson}(\delta * I_{\text{obs}}(t))\\
    I_{\text{obs}}(t+1) &= I_{\text{obs}}(t) + NI_{\text{obs}}(t+1) - NR_{\text{obs}}(t+1) - ND_{\text{obs}}(t+1)\\
    R_{\text{obs}}(t+1) &= R_{\text{obs}}(t) + NR_{\text{obs}}(t)\\
    D_{\text{obs}}(t+1) &= D_{\text{obs}}(t) + ND_{\text{obs}}(t)
\end{align}

We assume that $b$ and $q$ change day to day bases while other parameters are fixed.
To get a reasonable estimate, we assume prior distributions somewhat arbitrary chosen.
\begin{gather}
    S(0) = I(0) \sim \textup{Gamma}(1, 1)\\
    b(0) \sim \textup{Gamma}(1, 1)\\
    q(0) \sim \textup{Beta}(1, 1)\\
    b(t+1) \sim \textup{Gamma}(b(t), 1)\\
    q(t+1) \sim \textup{Beta}(q(t), 1-q(t))
\end{gather}

\subsection{Implementation}

\bibliographystyle{plain}
\bibliography{BayesianCOVID-19}

\end{document}